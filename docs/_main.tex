% Options for packages loaded elsewhere
\PassOptionsToPackage{unicode}{hyperref}
\PassOptionsToPackage{hyphens}{url}
%
\documentclass[
]{book}
\usepackage{amsmath,amssymb}
\usepackage{lmodern}
\usepackage{iftex}
\ifPDFTeX
  \usepackage[T1]{fontenc}
  \usepackage[utf8]{inputenc}
  \usepackage{textcomp} % provide euro and other symbols
\else % if luatex or xetex
  \usepackage{unicode-math}
  \defaultfontfeatures{Scale=MatchLowercase}
  \defaultfontfeatures[\rmfamily]{Ligatures=TeX,Scale=1}
\fi
% Use upquote if available, for straight quotes in verbatim environments
\IfFileExists{upquote.sty}{\usepackage{upquote}}{}
\IfFileExists{microtype.sty}{% use microtype if available
  \usepackage[]{microtype}
  \UseMicrotypeSet[protrusion]{basicmath} % disable protrusion for tt fonts
}{}
\makeatletter
\@ifundefined{KOMAClassName}{% if non-KOMA class
  \IfFileExists{parskip.sty}{%
    \usepackage{parskip}
  }{% else
    \setlength{\parindent}{0pt}
    \setlength{\parskip}{6pt plus 2pt minus 1pt}}
}{% if KOMA class
  \KOMAoptions{parskip=half}}
\makeatother
\usepackage{xcolor}
\IfFileExists{xurl.sty}{\usepackage{xurl}}{} % add URL line breaks if available
\IfFileExists{bookmark.sty}{\usepackage{bookmark}}{\usepackage{hyperref}}
\hypersetup{
  pdftitle={Small Handbook of Asset Pricing Essentials},
  pdfauthor={Sangmin S. Oh},
  hidelinks,
  pdfcreator={LaTeX via pandoc}}
\urlstyle{same} % disable monospaced font for URLs
\usepackage{longtable,booktabs,array}
\usepackage{calc} % for calculating minipage widths
% Correct order of tables after \paragraph or \subparagraph
\usepackage{etoolbox}
\makeatletter
\patchcmd\longtable{\par}{\if@noskipsec\mbox{}\fi\par}{}{}
\makeatother
% Allow footnotes in longtable head/foot
\IfFileExists{footnotehyper.sty}{\usepackage{footnotehyper}}{\usepackage{footnote}}
\makesavenoteenv{longtable}
\usepackage{graphicx}
\makeatletter
\def\maxwidth{\ifdim\Gin@nat@width>\linewidth\linewidth\else\Gin@nat@width\fi}
\def\maxheight{\ifdim\Gin@nat@height>\textheight\textheight\else\Gin@nat@height\fi}
\makeatother
% Scale images if necessary, so that they will not overflow the page
% margins by default, and it is still possible to overwrite the defaults
% using explicit options in \includegraphics[width, height, ...]{}
\setkeys{Gin}{width=\maxwidth,height=\maxheight,keepaspectratio}
% Set default figure placement to htbp
\makeatletter
\def\fps@figure{htbp}
\makeatother
\setlength{\emergencystretch}{3em} % prevent overfull lines
\providecommand{\tightlist}{%
  \setlength{\itemsep}{0pt}\setlength{\parskip}{0pt}}
\setcounter{secnumdepth}{5}
\usepackage{booktabs}
\usepackage{amsmath}
\ifLuaTeX
  \usepackage{selnolig}  % disable illegal ligatures
\fi
\usepackage[]{natbib}
\bibliographystyle{plainnat}

\title{Small Handbook of Asset Pricing Essentials}
\author{Sangmin S. Oh}
\date{}

\begin{document}
\maketitle

{
\setcounter{tocdepth}{1}
\tableofcontents
}
\hypertarget{introduction}{%
\chapter{Introduction}\label{introduction}}

This handbook, conveniently abbreviated as SHAPE, is designed to facilitate the review of key ideas and concepts necessary for asset pricing research. Much of the material is based primarily on my coursework at Chicago Booth, independent follow-ups, and discussions with my peers.

\hypertarget{part-classical-ap-essentials}{%
\part*{CLASSICAL AP ESSENTIALS}\label{part-classical-ap-essentials}}
\addcontentsline{toc}{part}{CLASSICAL AP ESSENTIALS}

\hypertarget{portfolio-choice}{%
\chapter{Portfolio Choice}\label{portfolio-choice}}

\hypertarget{mean-variance}{%
\section{Mean-Variance}\label{mean-variance}}

\textbf{CARA-Normal Framework}

One useful benchmark is what is referred to as the CARA-Normal framework, in which the agent has CARA utility and the risky asset has returns that are normally distributed. While tractable, a key disadvantage of this approach is that wealth does not affect the amount invested in a risky asset.

\textbf{CRRA Utility} \textbf{and Mean-Variance}

Instead, we now first assume that returns are lognormally distributed. It's appealing to assume that returns are lognormal in a discrete-time model with IID returns because returns will then be lognormal over multiple periods as well. Furthermore, let's introduce CRRA utility:
\[
\max \left[\mathbb{E}_t \frac{W_{t+1}^{1-\gamma}}{1-\gamma}\right]
\]
Then we can show that the investor indeed trades off mean against variance if we assume that investor's wealth is lognormally distributed. To see this, rewrite above as
\[
\max[\log\mathbb{E}_tW_{t+1}^{1-\gamma}] = \max\left[(1-\gamma) \mathbb{E}_tw_{t+1} + \frac{1}{2}(1-\gamma)^2 \sigma_{wt}^2\right]
\]
where \(w_t =\log W_t\) and \(\sigma_{wt}^2\) is the conditional variance of log wealth. We can then use the budget constraint \(w_{t+1} = r_{p,t+1}+w_t\) to restate the problem:
\[
\max \left[\mathbb{E}_t r_{p,t+1} + \frac{1}{2} (1-\gamma) \sigma_{pt}^2\right]=\max\left[\log \mathbb{E}_t r_{p,t+1} - \frac{\gamma}{2} \sigma_{pt}^2\right]
\]
Thus the investor trades the log of the arithmetic mean return linearly against the variance of the log return.

\textbf{Mutual Fund Theorem}

The mutual fund theorem of Tobin (1958) says that all minimum-variance portfolios can be obtained by mixing just two minimum-variance portfolios in different proportions. Thus, if all investors hold minimum-variance portfolios, all investors hold combinations of just two underlying portfolios or ``mutual funds.''

In the presence of a riskless asset, the mutual fund theorem simplifies because one of the mutual funds is the riskless asset and the other, the tangency portfolio, contains only risky assets. Thus it says that all investor, regardless of their risk aversion, should hold risky assets in the same proportion.

Interpreting bonds and stocks as two alternative risky assets, the mutual fund theorem implies that the ratio of bonds to stocks should be constant across recommended portfolios, while the ratio of cash to (bonds + stocks) should move in the same direction as risk aversion. Canner, Mankiw, and Weil (1997) find that the ratio of bonds to stocks moves with risk aversion.

\hypertarget{expected-return---beta-representations}{%
\section{Expected Return - Beta Representations}\label{expected-return---beta-representations}}

\textbf{Price and Quantity of Risk}

It's common to consider the representation of the following form:
\[
\mathbb{E}[R^i] = \alpha + \beta_{i,a}\lambda_a + \beta_{i,b}\lambda_b +\cdots
\]
It says that assets with higher betas should get higher average returns. \(\beta_{i,a}\), often referred to as quantity of risk, is interpreted as the amount of exposure of asset \(i\) to factor \(a\) risks, and \(\lambda_a\) is the \textbf{price of risk}. In other words, for each unit of exposure \(\beta\) to risk factor \(a\), you must provide investors with an expected return premium \(\lambda_a\).

\hypertarget{stochastic-discount-factor}{%
\chapter{Stochastic Discount Factor}\label{stochastic-discount-factor}}

\hypertarget{sdf-basics}{%
\section{SDF Basics}\label{sdf-basics}}

\textbf{Existence of SDFs}

There are two key theorems regarding the existence of an SDF:

\begin{enumerate}
\def\labelenumi{\arabic{enumi}.}
\tightlist
\item
  There is a discount factor that prices all the payoffs by \(P=\mathbb{E}[MX]\) if and only if the law of one price holds.
\item
  There is a \emph{positive} discount factor that prices all the payoffs by \(P = \mathbb{E}[MX]\) if and only if there are no arbitrage opportunities.
\end{enumerate}

These theorems are useful to show that we can use SDFs without implicitly assuming anything about utility functions, aggregate, and market completeness.

\textbf{Fundamental Equation of Asset Pricing}

Law of one price implies that we must have
\[
P(X) = \sum_{s=1}^S q(s)X(s) = \sum_{s=1}^S \pi(s)\left(\frac{q(s)}{\pi(s)}\right)X(s) = \mathbb{E}[MX]
\]
where \(M(s) = q(s) / \pi(s)\) is the ratio of state price to probability for state \(s\).

\textbf{Risk-Neutral Pricing}

We can further rewrite the above equation by defining risk-neutral probabilities as \(\pi^*(s)\):
\[
\pi^*(s) = R_f q(s) =\frac{M(s)}{\mathbb{E}[M]} \pi(s)
\]
which allows us to rewrite the asset pricing equation:
\[
P(X) = \frac{1}{R_f} \sum_s=1^S \pi^*(s) X(s) = \frac{1}{R_f} \mathbb{E}^*[X]
\]
So the price of any asset is the pseudo-expectation of its payoff, discounted at the riskless interest rate.

\hypertarget{properties-of-sdfs}{%
\subsection{Properties of SDFs}\label{properties-of-sdfs}}

\textbf{Linear Factor Pricing Model}

We can show that returns always obey a linear factor pricing model with the SDF as the single factor. To see this, start with the fundamental equation of asset pricing:
\[
P_{it} = \mathbb{E}_t[M_{t+1}X_{i,t+1}] = \mathbb{E}_t[M_{t+1}]\mathbb{E}_t[X_{i,t+1}] + Cov_t(M_{t+1}, X_{i,t+1})
\]
Dividing each side by \(P_{it}\) and using the fact that \(1+R_{f,t+1} = 1 / \mathbb{E}_t[M_{t+1}]\) yields:
\[
\mathbb{E}_t[1+R_{i,t+1}] = (1 + R_{f,t+1})(1 - Cov_t(M_{t+1}, R_{i,t+1}))
\]
which says that expected return on any asset is the riskless return times an adjustment factor for the covariance of return with the SDF.

As a final step, we can subtract the gross risk-free rate from each side:
\[
\begin{aligned}
\mathbb{E}_t[R_{i,t+1} - R_{f,t+1}] &= -(1+R_{f,t+1}) Cov_t(M_t+1, R_{i,t+1} - R_{f,t+1})\\
                                    &= \underbrace{-(1+R_{f,t+1})Var_t(M_{t+1})}_{\equiv \lambda_{Mt}}\underbrace{\frac{Cov_t(M_t+1, R_{i,t+1} - R_{f,t+1})}{Var_t(M_{t+1})}}_{\equiv \beta_{iMt}}
\end{aligned}
\]
We denote \(\lambda_{Mt}\) as the price of risk or the factor risk premium of the SDF. Immediately, we see that it depends on the volatility of the SDF.

\textbf{Deriving the Hansen-Jagannathan Bound}

Hansen--Jagannathan bound, introduced in \protect\hyperlink{HansJaga:91}{Hansen and Jagannathan (1991)} is a theorem that says that the ratio of the standard deviation of a stochastic discount factor to its mean exceeds the Sharpe ratio attained by any portfolio. Deriving the bound with a risky and a riskless asset is easy. Specifically, write:
\[
\begin{aligned}
\mathbb{E}_t[R_{i,t+1} - R_{f,t+1}] &= -\frac{Cov_t(M_{t+1}, R_{i,t+1} - R_{f,t+1})}{\mathbb{E}_t[M_{t+1}]}\\
                                    &\leq \frac{\sigma_t(M_{t+1})\sigma_t(R_{i,t+1} - R_{f,t+1})}{\mathbb{E}_t[M_{t+1}]}
\end{aligned}
\]
Rearranging therefore yields:
\[
\frac{\sigma_t(M_{t+1})}{\mathbb{E}_t[M_{t+1}]} \geq \frac{\mathbb{E}_t[R_{i,t+1} - R_{f,t+1}]}{\sigma_t(R_{i,t+1} - R_{f,t+1})}
\]
Hansen and Jagannathan (1991) also derive the bound even when there is no riskfree asset pinning down the mean of the SDF. The idea is to treat the mean of the SDF as an unknown parameter, and for each possible value of the mean, augment the set of basis assets with a hypothetical riskfree payoff whose return equals \(1/\bar{M}\).

\textbf{Usefulness of the Hansen-Jagannathan Bound}

The HJ frontier is commonly used as a quick check on the ability of a parametric asset pricing model to fit the properties of asset returns. The mean and volatility of the SDF can be calculated for different parameter values of the model, and if they fail to satisfy the SDF volatility bounds, then this indicates that the model fails to price the assets.

\begin{itemize}
\tightlist
\item
  For example, Hansen and Jagannathan (1991) calculate SDF volatility bounds using return data on Treasury bills and an aggregate stock index. They find that a simple consumption-based asset pricing model with a power-utility representative agent can only satisfy these bounds if very high risk aversion coefficients are used.
\end{itemize}

\hypertarget{present-value-relations}{%
\chapter{Present Value Relations}\label{present-value-relations}}

\hypertarget{dynamic-factor-models}{%
\chapter{Dynamic Factor Models}\label{dynamic-factor-models}}

\hypertarget{estimating-and-evaluating-models}{%
\chapter{Estimating and Evaluating Models}\label{estimating-and-evaluating-models}}

\hypertarget{paper-highlights}{%
\chapter{Paper Highlights}\label{paper-highlights}}

\hypertarget{HansJaga:91}{%
\subsection{Hansen and Jagannathan (1991)}\label{HansJaga:91}}

This paper\ldots{}

\hypertarget{part-shoulders-of-giants}{%
\part*{SHOULDERS OF GIANTS}\label{part-shoulders-of-giants}}
\addcontentsline{toc}{part}{SHOULDERS OF GIANTS}

\hypertarget{epstein-zin-preferences-and-long-run-risks}{%
\chapter{Epstein-Zin Preferences and Long-Run Risks}\label{epstein-zin-preferences-and-long-run-risks}}

\hypertarget{incomplete-markets}{%
\chapter{Incomplete Markets}\label{incomplete-markets}}

\hypertarget{rare-events-and-disasters}{%
\chapter{Rare Events and Disasters}\label{rare-events-and-disasters}}

\hypertarget{habit-formation}{%
\chapter{Habit Formation}\label{habit-formation}}

\hypertarget{ambiguity-aversion}{%
\chapter{Ambiguity Aversion}\label{ambiguity-aversion}}

\hypertarget{learning}{%
\chapter{Learning}\label{learning}}

\hypertarget{production-based-models}{%
\chapter{Production-based Models}\label{production-based-models}}

\hypertarget{term-structure}{%
\chapter{Term Structure}\label{term-structure}}

\hypertarget{part-pricing-specific-assets}{%
\part*{PRICING SPECIFIC ASSETS}\label{part-pricing-specific-assets}}
\addcontentsline{toc}{part}{PRICING SPECIFIC ASSETS}

\hypertarget{pricing-currencies}{%
\chapter{Pricing Currencies}\label{pricing-currencies}}

\hypertarget{pricing-volatility}{%
\chapter{Pricing Volatility}\label{pricing-volatility}}

\hypertarget{pricing-corporate-bonds}{%
\chapter{Pricing Corporate Bonds}\label{pricing-corporate-bonds}}

\hypertarget{pricing-government-bonds}{%
\chapter{Pricing Government Bonds}\label{pricing-government-bonds}}

\hypertarget{pricing-equity-strips}{%
\chapter{Pricing Equity Strips}\label{pricing-equity-strips}}

\hypertarget{part-selected-topics}{%
\part*{SELECTED TOPICS}\label{part-selected-topics}}
\addcontentsline{toc}{part}{SELECTED TOPICS}

\hypertarget{asset-pricing-around-announcements}{%
\chapter{Asset Pricing around Announcements}\label{asset-pricing-around-announcements}}

\hypertarget{machine-learning-in-asset-pricing}{%
\chapter{Machine Learning in Asset Pricing}\label{machine-learning-in-asset-pricing}}

\hypertarget{demand-system-approach}{%
\chapter{Demand System Approach}\label{demand-system-approach}}

\hypertarget{subjective-beliefs-in-asset-pricing}{%
\chapter{Subjective Beliefs in Asset Pricing}\label{subjective-beliefs-in-asset-pricing}}

\hypertarget{insurance}{%
\chapter{Insurance}\label{insurance}}

\hypertarget{international-finance}{%
\chapter{International Finance}\label{international-finance}}

\hypertarget{part-monetary-policy}{%
\part*{MONETARY POLICY}\label{part-monetary-policy}}
\addcontentsline{toc}{part}{MONETARY POLICY}

\hypertarget{basics-of-new-keynesian-framework}{%
\chapter{Basics of New Keynesian Framework}\label{basics-of-new-keynesian-framework}}

\hypertarget{monetary-policy-shocks}{%
\chapter{Monetary Policy Shocks}\label{monetary-policy-shocks}}

\hypertarget{central-banks-and-asset-prices}{%
\chapter{Central Banks and Asset Prices}\label{central-banks-and-asset-prices}}

  \bibliography{book.bib,packages.bib}

\end{document}
